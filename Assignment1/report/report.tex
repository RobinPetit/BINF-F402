%\documentclass{IEEEtran}
\documentclass{article}

\usepackage[utf8]{inputenc}
\usepackage[T1]{fontenc}
\usepackage[cm]{fullpage}

\usepackage[hidelinks]{hyperref}
\usepackage{breakurl}  % break urls at '/' if too long for \textwidth
\usepackage{caption}
\usepackage{subcaption}

\usepackage{graphicx}
\usepackage{float}
\usepackage{listings}
\usepackage{color}
\usepackage[parfill]{parskip}
\usepackage{commath}

\definecolor{mygreen}{rgb}{0,0.6,0}
\definecolor{mygray}{rgb}{0.5,0.5,0.5}
\definecolor{mymauve}{rgb}{0.58,0,0.82}

\lstset{language=R}

\lstset{ %
  backgroundcolor=\color{white},   % choose the background color; you must add \usepackage{color} or \usepackage{xcolor}; should come as last argument
  basicstyle=\footnotesize,        % the size of the fonts that are used for the code
  breakatwhitespace=false,         % sets if automatic breaks should only happen at whitespace
  breaklines=true,                 % sets automatic line breaking
  captionpos=b,                    % sets the caption-position to bottom
  commentstyle=\color{mygreen},    % comment style
  deletekeywords={...},            % if you want to delete keywords from the given language
  escapeinside={\%*}{*)},          % if you want to add LaTeX within your code
  extendedchars=true,              % lets you use non-ASCII characters; for 8-bits encodings only, does not work with UTF-8
  frame=single,	                   % adds a frame around the code
  keepspaces=true,                 % keeps spaces in text, useful for keeping indentation of code (possibly needs columns=flexible)
  keywordstyle=\color{blue},       % keyword style
  numbers=left,                    % where to put the line-numbers; possible values are (none, left, right)
  numbersep=4pt,                   % how far the line-numbers are from the code
  numberstyle=\tiny\color{mygray}, % the style that is used for the line-numbers
  rulecolor=\color{black},         % if not set, the frame-color may be changed on line-breaks within not-black text (e.g. comments (green here))
  showspaces=false,                % show spaces everywhere adding particular underscores; it overrides 'showstringspaces'
  showstringspaces=false,          % underline spaces within strings only
  showtabs=false,                  % show tabs within strings adding particular underscores
  stepnumber=1,                    % the step between two line-numbers. If it's 1, each line will be numbered
  stringstyle=\color{mymauve},     % string literal style
  tabsize=2,	                     % sets default tabsize to 2 spaces
	gobble=1,
}

\title{BINFF-402 --- Genomics, Proteomics and Evolution \\ Assignment 1}
\author{Robin Petit}

\begin{document}
\maketitle

\section{Introduction}

The Illumina Infinium 450K sequencer~\cite{infinium450} was designed in order to sequence DNA methylation. It works by first applying
bisulfite transformation to the DNA fragments to transform unmethylated Cytosines to Uracils, and leave methylated Cytosines unchanged,
and then sequencing uses two different bead types: one to determine methylated loci and one to determine unmethylated
loci.~\cite{weisenberger2008comprehensive}

The Infinium HumanMethylation450 BeadChip contains over 4.5e5 methylation sites probes, and two different chemical assays: Infinium I and
Infinium II. In this assignment, only Infinium I is considered. Yet, Infinium II should also be considered in analyses, but
separately.~\cite{dedeurwaerder2011evaluation}

In this document, methylation level is studied in the case of a DKO of genes DNMT1 and DNMT3b (respectively OMIM ids 126375 and 602900)~\cite{amberger2008OMIM}
which are both methyltransferases. This double knockout induces a loss of hypermethylated CpG islands which is not present in case of
single knockout of either of these genes.~\cite{paz2003genetic}

The study has been performed with the R language. Note that all code has been written from scratch even though there is a wide range of
available packages for bioinformatics in R~\cite{gentleman2004bioconductor}. See for instance the package IMA that is made on purpose
for Infinium data analysis~\cite{wang2012ima}.

All the R code used for this report can be found at~\url{https://github.com/RobinPetit/BINF-F402}.

\section{Analysis}
\subsection{$\beta$-value distribution}

\begin{figure*}[!t]
	\begin{subfigure}{.33\textwidth}
	\includegraphics[width=\textwidth]{../plots/plot0.eps}
	\end{subfigure}
	\begin{subfigure}{.33\textwidth}
	\includegraphics[width=\textwidth]{../plots/plot1.eps}
	\end{subfigure}
	\begin{subfigure}{.33\textwidth}
	\includegraphics[width=\textwidth]{../plots/plot2.eps}
	\end{subfigure}
	\begin{subfigure}{.33\textwidth}
	\includegraphics[width=\textwidth]{../plots/plot3.eps}
	\end{subfigure}
	\begin{subfigure}{.33\textwidth}
	\includegraphics[width=\textwidth]{../plots/plot4.eps}
	\end{subfigure}
	\begin{subfigure}{.33\textwidth}
	\includegraphics[width=\textwidth]{../plots/plot5.eps}
	\end{subfigure}
	\caption{Distribution of the $\beta$ value for each sample. Samples 1 to 3 are the controls, and samples 4 to 6 are the cases.
	\label{fig:beta-distribution}}
\end{figure*}

The $\beta$ distribution is shown in~\autoref{fig:beta-distribution}. For each probe, the $\beta$ value is computed to be $\frac M{M+U+\alpha}$,
where $M$ is the methylated score, $U$ is the unmethylated score, and $\alpha$ acts as a pseudo-count to avoid the case $M=U=0$ to lead
to a division by zero. This $\alpha$ value has been set to 100.~\cite{du2010comparison}

It is clear that controls have a two-spikes $\beta$ distribution: most probes have a methylation $\beta$-value around 0 or around 1.
A high methylation $\beta$-value (close to 1) means that almost all of the sequenced cells had a methylated Cytosine at this position,
whereas a low methylation $\beta$-value (close to 0) means that almost none of the sequenced cells had a methylated Cytosine at this
position. Yet, as several different cells are sequenced at the same time, it is possible to have a Cytosine that is methylated in
some cells, but not in others. This explains how $\beta$-value can take so many different values.

It is also clear that cases have a single-spike $\beta$ distribution: most probes have a methylation $\beta$-value close to 0,
but almost none has a $\beta$-value close to 1. Also, there is a higher density of intermediate $\beta$-values, i.e. between $0.2$
and $0.8$, which means that there is still methylation happening in the cases, but the methylation of DNA varies from one cell
to another.

\subsection{Global overview}

An analysis of the methylation level of each chromosome is shown in~\autoref{fig:methylation all chromosome}. We can observe that the
hypomethylation is present on each chromosome, with highest hypomethylation being on chromosomes 10, 16 and 20 and lowest hypomethylation
is on chromosome Y, probably because the Y chromosome has a lower methylation level in control samples. Yet, chromosome X is also
less hypomethylated than the other chromosomes, but sum of difference in methylation level for X and Y chromosomes reaches roughly
the same value as the other chromosomes.

The distribution of the difference in methylation level per chromosome seems to be pretty uniform between $0.2$ and $0.25$.

\autoref{fig:methylation 19-21} shows the methylation level of chromosomes 19 to 21 as stated in the instructions.

\begin{figure}[!t]
	\begin{subfigure}{.5\textwidth}
		\includegraphics[width=\textwidth]{../plots/plot6.eps}
		\subcaption{Methylation level on each chromosome (1 to 22 plus X and Y).\label{fig:methylation all chromosome}}
	\end{subfigure}
	\begin{subfigure}{.5\textwidth}
		\includegraphics[width=\textwidth]{../plots/plot7.eps}
		\subcaption{Subplot of~\autoref{fig:methylation per chromosome} of chromosomes 19 to 21.\label{fig:methylation 19-21}}
	\end{subfigure}
	\caption{General analysis of the methylation level per chromosome for both controls and cases samples. Difference in methylation
	is defined as control minus case since cases are hypomethylated.\label{fig:methylation per chromosome}}
\end{figure}

\subsection{Number of probes per gene}

The distribution of the number of probes per gene is shown in~\autoref{fig:number of probes per gene}:~\autoref{fig:histogram of nb probes per gene}
shows the distribution in which we observe that almost all of the genes have only a few probes related to them, so~\autoref{fig:number of probes per gene}
shows the same data in y log-scale.

Yet, as displayed on~\autoref{fig:log plot nb probes per gene}, the average number of probes per gene is around $32.36$. This mean value is
impacted by the very high values that are 2199, 2404, and 3897, but since only one such gene exist for each of these values, the impact is
very low, due to the big amount of probes: 485577 (computed by either \texttt{nrow(infinium)} in the R code, or by \texttt{wc -l annotations.csv}
minus 1 to remove the header).

\begin{figure}[!b]
	\begin{subfigure}{.5\textwidth}
		\includegraphics[width=\textwidth]{../plots/plot8.eps}
		\subcaption{Distribution of the number of probes per gene.\label{fig:histogram of nb probes per gene}}
	\end{subfigure}
	\begin{subfigure}{.5\textwidth}
		\includegraphics[width=\textwidth]{../plots/plot9.eps}
		\caption{Log-distribution of the number of probes per gene.\label{fig:log plot nb probes per gene}}
	\end{subfigure}
	\caption{Distribution of the number of probes per gene in both linear and log scale.\label{fig:number of probes per gene}}
\end{figure}

\subsection{$\Delta\beta$}

\begin{figure}[!b]
	\begin{subfigure}{.5\textwidth}
		\includegraphics[width=\textwidth]{../plots/plot10.eps}
	\end{subfigure}
	\begin{subfigure}{.5\textwidth}
		\includegraphics[width=\textwidth]{../plots/plot11.eps}
	\end{subfigure}
	\begin{subfigure}{.5\textwidth}
		\includegraphics[width=\textwidth]{../plots/plot12.eps}
	\end{subfigure}
	\begin{subfigure}{.5\textwidth}
		\includegraphics[width=\textwidth]{../plots/plot13.eps}
	\end{subfigure}
	\caption{Distribution of $\Delta\beta$ for each couple control/case and in average.\label{fig:delta beta}}
\end{figure}

The distribution of $\Delta\beta$ (control - case) is shown in~\autoref{fig:delta beta}. The three first subplots represent the $\Delta\beta$
distribution of each couple (control, case), but the most interesting part is the last subplot showing the $\Delta\beta$ in average for each
control and each case sample. We observe no hypermethylation ($\Delta\beta > 0$) but a clear hypomothylation ($\Delta\beta < 0$) as expected.

Still, most of the probability mass is around $\Delta\beta = 0$. This comes from the fact that even though case samples are hypomethylated,
they still contain methlyated regions (see~\autoref{fig:beta-distribution} and \autoref{fig:methylation per chromosome}),
for which $\Delta\beta$ is close to 0.

\subsection{$p$-values and volcano plot}

$p$-value of the $\Delta\beta$ value for each probe have been computed by performing a Student $t$-test. Their distribution is shown
in~\autoref{fig:p value distribution}. As can be seen on~\autoref{fig:distribution of log p value}, many probes have high $p$-value (around
$10^{\frac {-1}2}$) which is far from significant, but still a big part is below $10^{-3}$.

Note that a $t$-test takes variance into consideration, meaning that a high $\Delta\beta$ might not be significant if variance is high
inside the groups.

\begin{figure}[!h]
	\begin{subfigure}{.5\textwidth}
		\includegraphics[width=\textwidth]{../plots/plot14.eps}
		\caption{Distribution of $p$-value.\label{fig:distribution of p values}}
	\end{subfigure}
	\begin{subfigure}{.5\textwidth}
		\includegraphics[width=\textwidth]{../plots/plot15.eps}
		\caption{Distribution of base 10 log $p$-value.\label{fig:distribution of log p value}}
	\end{subfigure}
	\caption{Distribution of $p$-value and $\log_{10}(p$-value$)$.\label{fig:p value distribution}}
\end{figure}

\autoref{fig:volcano} plots $\Delta\beta$ and $p$-values in a volcano plot. We can observe a big portion of the probes in the upper
right area of the plot, meaning significant $\Delta\beta$ of hypomethylation, but we can also observe the upper left area of the plot
showing significant $\Delta\beta$ of hypermethylation.

\begin{figure}
	\centering
	\includegraphics[width=\textwidth]{../plots/plot16.eps}
	\caption{Volcano plot of $p$-value according to $\Delta\beta$. Significance threshold for $p$-values is set to $p = 10^{-3}$, and for
	$\abs {\Delta\beta} = .2$. Note that even though the study here is about hypomethylated cases ($\Delta\beta > 0$), several probes lie
	before $\Delta\beta = -.2$, meaning that some regions are hypermethylated (even \textit{significantly} hypermethylated).
	\label{fig:volcano}}
\end{figure}

\subsection{Heatmap}

When taking the top 30 probes according to $\abs {\Delta\beta}$, we can plot them in a heatmap (see~\autoref{fig:heatmap}).
Separate plots for methylated and unmethylated probes are shown in~\autoref{fig:heatmap separated}.

For the 3 first samples (i.e. the controls), we can see that the unmethylated probes received way less signals than the
methylated ones, and that this tendency is reversed for the 3 last samples (i.e. the cases). This is as expected since the
probes that are present in this plot are the ones with the highest $\Delta\beta$ value, meaning that the difference between
the methylation level in the controls and in the cases is the highest. Therefore, it is expected to have highly methylated
controls and highly unmethylated cases signals.

\begin{figure}
	\centering
	\includegraphics[width=.5\textwidth]{../plots/plot17.eps}
\caption{Heatmap of the probes value of the 30 probes having the highest average $\Delta\beta$.
\label{fig:heatmap}}
\end{figure}

\begin{figure}
	\begin{subfigure}{.5\textwidth}
		\includegraphics[width=\textwidth]{../plots/plot18.eps}
		\caption{Heatmap of the unmethylated probes value of the 30 probes having the highest $\Delta\beta$.
		\label{fig:heatmap unmethylated}}
	\end{subfigure}
	\begin{subfigure}{.5\textwidth}
		\includegraphics[width=\textwidth]{../plots/plot19.eps}
		\caption{Heatmap of the methylated probes value of the 30 probes having the highest $\Delta\beta$.
		\label{fig;heatmap methylated}}
	\end{subfigure}
	\caption{Separate heatmaps for methylated and unmethylated probes of the top 30 probes.\label{fig:heatmap separated}}
\end{figure}

% Second part
\newpage
\section{R source code}

First of all, libraries must be imported and constants must be defined. Only two libraries have been used files must be loaded into R:

\begin{lstlisting}
	library('latex2exp');  # Lib necessary to use LaTeX in plots
	library('gplots');     # for heatmap.2

	INFINIUM_PATH  <- 'Infinium450k_raw_data.txt';
	ANNO_MINI_PATH <- 'HumanMethylation450_anno_mini.csv';
	NB_CASES <- 3;
	NB_CONTROLS <- NB_CASES;
	CONTROLS <- 1:NB_CONTROLS;
	CASES <- (1:NB_CASES) + NB_CONTROLS;
	ALPHA_PSEUDO_COUNT <- 100;
	ALL_CHROMOSOMES <- c(1:22, 'X', 'Y');
	CHROMOSOMES_TO_TEST <- 19:21;

	LOG_P_VALUE_THRESHOLD <- -3;
	DELTA_BETA_THRESHOLD <- .2;

	# Use first column as row names
	infinium <- read.table(INFINIUM_PATH, header=T, dec=',', row.names=1);
	annotations <- read.table(ANNO_MINI_PATH, header=T, sep=',', row.names=1);
\end{lstlisting}

In order to determine the $\beta$ value of a sample, let's define a few functions:
\begin{lstlisting}
	compute.beta <- function(signalA, signalB) {
		# Note the pseudo count $\alpha = 100$ to avoid dividing by 0
		return (signalB / (signalA + signalB + ALPHA_PSEUDO_COUNT));
	}

	# Get the column corresponding to the required sample
	get.sample.signal <- function(table, sample.id, signalA=T) {
		if(signalA) {
			return (as.vector(table[[2*sample.id-1]]));
		} else {
			return (as.vector(table[[2*sample.id]]));
		}
	}

	# Returns the $\beta$ values of a given sample
	get.beta <- function(infinium, sample.id) {
		return (compute.beta(
			get.sample.signal(infinium, sample.id, T),
			get.sample.signal(infinium, sample.id, F)
		));
	}
\end{lstlisting}

so that plotting becomes:
\begin{lstlisting}
	# Plot the beta distribution of each sample from 1 to 6
	plot.beta.distribution.infinium <- function(infinium) {
		for(sample.id in 1:(NB_CASES+NB_CONTROLS)) {
			plot(
				density(get.beta(infinium, sample.id)),
				xlim=c(-.2, 1.2),
				ylim=c(0, 5),
				xlab=TeX('$\\beta$'),
				ylab=TeX('Density of $\\beta$'),
				main=TeX(paste('density of $\\beta$ distribution of sample', sample.id)),
			);
		}
	}
	plot.beta.distribution.infinium(infinium);
\end{lstlisting}

The methylation level per chromosome is performed as follows:
\begin{lstlisting}
	# Returns the subtable corresponding to the required chromosome
	extract.chromosome <- function(annotations, chr) {
		return (annotations[annotations$CHR == chr,]);
	}

	# Make the barplots of the average methylation level of each chromosome
	plot.mean.methylation.level <- function(infinium, annotations) {
		means <- matrix(rep(0, 3*length(ALL_CHROMOSOMES)), nrow=3, ncol=length(ALL_CHROMOSOMES));
		idx <- 1;
		for(chromosome in ALL_CHROMOSOMES) {
			intersection <- intersect(rownames(infinium), rownames(extract.chromosome(annotations, chromosome)));
			chromosome.infinium.subtable <- infinium[intersection,];
			for(case.sample.id in CASES)
				means[1,idx] <- means[1,idx] + mean(get.beta(chromosome.infinium.subtable, case.sample.id))/NB_CASES;
			for(control.sample.id in CONTROLS)
				means[2,idx] <- means[2,idx] + mean(get.beta(chromosome.infinium.subtable, control.sample.id))/NB_CONTROLS;
			# Take difference control - case to see methylation difference per chromosome
			means[3,] <- means[2,] - means[1,];
			idx <- idx+1;
		}
		barplot(means, names.arg=ALL_CHROMOSOMES, horiz=T, beside=T, legend=c('Cases', 'Controls', 'Difference'), las=1);
		barplot(means[,CHROMOSOMES_TO_TEST], names.arg=CHROMOSOMES_TO_TEST, beside=T, legend=c('Cases', 'Controls', 'Difference'), las=1);
	}


	plot.mean.methylation.level(infinium, annotations);
\end{lstlisting}

To find the number of probes per gene, only annotations are required since all ids are common to the infinium file and the annotations file:
\begin{lstlisting}
	# Plot the histograms of the distribution and log-distribution of the number of probes per gene
	plot.number.of.genes.per.probe <- function(annotations) {
		non.null.genes.associations <- as.vector(annotations[annotations$UCSC_RefGene_Name != '',]$UCSC_RefGene_Name);
		genes.counter <- table(unlist(sapply(non.null.genes.associations, function(s) {unique(strsplit(s, ';'))})));
		hist(genes.counter, main='Distribution of the number of probes per gene',
			sub=sprintf('Average number of probes per gene is %g', mean(genes.counter)),
			xlab='Number of probes per gene',
			ylab='Frequency');
		plot(log(table(genes.counter)), main='log plot of the distribution of the number of probes per gene',
			xlab='Number of probes per gene',
			ylab='log(Frequency)');
	}
\end{lstlisting}

Note that \texttt{function(s) {unique(strsplit(s, ';'))}} allows to retrieve all the genes associated to a probe, but not taking
the genes that come several times, thanks to the \texttt{unique} function. Making a table of the unlisted results (since
\texttt{sapply} returns a list) makes R count the number of probes related to each gene on its own. Therefore, making a histogram
of the table is sufficient to plot the distribution.

$\Delta\beta$ distribution is computed and plotted as follows:
\begin{lstlisting}
	# Get the $\Delta\beta$ vector of a (control, sample) couple
	get.delta.beta <- function(infinium, case.sample.id, control.sample.id) {
		return (get.beta(infinium, control.sample.id) - get.beta(infinium, case.sample.id));
	}

	# Plots the density of $\Delta\beta$ for each (control, sample) couple
	plot.delta.beta.distribution <- function(infinium) {
		# plot \Delta \beta distribution for each couple control/case
		for(control.sample.id in CONTROLS) {
			case.sample.id <- control.sample.id + NB_CONTROLS;
			infinium$de
			plot(
				density(get.delta.beta(infinium, case.sample.id, control.sample.id)),
				main=TeX(sprintf('$\\Delta\\beta$ (control - case) for couple #%d', control.sample.id)),
				xlab=TeX('$\\Delta\\beta$'),
				ylab='Density',
				xlim=c(-1, 1),
				ylim=c(0, 5.5)
			);
		}
		# Plot \Delta \beta for the mean of all samples
		delta.betas <- matrix(0, nrow=nrow(infinium), ncol=NB_CONTROLS+NB_CASES);
		for(control.sample.id in CONTROLS) {
			delta.betas[,control.sample.id] <- get.beta(infinium, control.sample.id);
			case.sample.id <- control.sample.id + NB_CONTROLS;
			delta.betas[,case.sample.id] <- get.beta(infinium, case.sample.id);
		}
		infinium$delta.beta <- apply(delta.betas[,CONTROLS], 1, mean) - apply(delta.betas[,CASES], 1, mean);
		plot(
			density(infinium$delta.beta),
			main=TeX('mean $\\Delta\\beta$ (controls - cases)'),
			xlab=TeX('$\\Delta\\beta$'),
			ylab='Density',
			xlim=c(-1, 1),
			ylim=c(0, 5.5)
		);
		infinium$p.values <- apply(delta.betas, 1,
			function(row) {
				return (t.test(row[CONTROLS], row[CASES])$p.value);
			}
		);
		infinium$log.p.value <- log(infinium$p.value, 10);
		plot(
			density(infinium$p.values),
			main=TeX('$p$-value density'),
			xlab=TeX('$p$-value'),
			ylab='density'
		);
		plot(
			density(-infinium$log.p.value),
			main=TeX('$-\\log_{10}(p$-value$)$ density'),
			xlab=TeX('$-\\log_{10}(p$-value$)$'),
			ylab='density'
		);
		return (infinium);
	}

	infinium <- plot.delta.beta.distribution(infinium);
\end{lstlisting}

Note that as \texttt{plot.delta.beta.distribution} returns the \texttt{infinium} table since it has been modified
and R does not allow passing parameters by reference. What is added to the table is the $p$-value of $\Delta\beta$
for each probe (computed by a $t$-test), that is also plotted.

With these $p$-values, the volcano plot is performed using:
\begin{lstlisting}
	plot.volcano <- function(infinium) {
		significant.idx <- as.logical((abs(infinium$delta.beta) >= .2) * (infinium$log.p.value <= -3));
		significant <- infinium[significant.idx,];
		non.significant <- infinium[!significant.idx,];
		plot(
			c(non.significant$delta.beta, significant$delta.beta),
			-c(non.significant$log.p.value, significant$log.p.value),
			col=c(rep('black', nrow(non.significant)), rep('red', nrow(significant))),
			xlab=TeX('$\\Delta\\beta$'),
			ylab=TeX('$-\\log_{10}(p$-value$)$')
		)
		lines(c(min(infinium$delta.beta), max(infinium$delta.beta)), rep(-LOG_P_VALUE_THRESHOLD, 2), col='grey', lty=2)
		lines(rep(DELTA_BETA_THRESHOLD, 2), c(min(-infinium$log.p.value), max(-infinium$log.p.value)), col='grey', lty=2)
		lines(rep(-DELTA_BETA_THRESHOLD, 2), c(min(-infinium$log.p.value), max(-infinium$log.p.value)), col='grey', lty=2)
	}

	plot.volcano(infinium);
\end{lstlisting}

Only the heatmaps are still missing. These are plotted as follows:
\begin{lstlisting}
	get.top.delta.beta <- function(infinium, n=30) {
		return (infinium[order(abs(infinium$delta.beta), decreasing=T)[1:n],]);
	}

	plot.heatmap <- function(infinium) {
		top <- get.top.delta.beta(infinium);
		par(mar=c(0, 1, 3, 1))
		heatmap.2(t(as.matrix(top[,1:(2*(NB_CONTROLS+NB_CASES))])), Rowv=NA, Colv=NA, scale='none', cexRow=.8, cexCol=.75,
			margins=c(6, 8), dendogram='none', trace='none');
		for(i in 1:2) {
			heatmap.2(t(as.matrix(top[,seq(i, 2*(NB_CONTROLS+NB_CASES),2)])), Rowv=NA, Colv=NA, scale='none', cexRow=.8, cexCol=.75,
				margins=c(6, 8), dendogram='none', trace='none');
		}
	}

	plot.heatmap(infinium);
\end{lstlisting}

Note that \texttt{heatmap.2} is used, which explains the \texttt{library('gplots');} above.

% references
\newpage
\bibliographystyle{IEEEtran}
\bibliography{report}{}

\end{document}
