%\documentclass{IEEEtran}
\documentclass{article}

\usepackage[utf8]{inputenc}
\usepackage[T1]{fontenc}
\usepackage{fullpage}

\usepackage{hyperref}
\usepackage{breakurl}  % break urls at '/' if too long for \textwidth
\usepackage{caption}
\usepackage{subcaption}

\usepackage{graphicx}
\usepackage{float}
\usepackage{listings}
\usepackage{color}
\usepackage[parfill]{parskip}

\definecolor{mygreen}{rgb}{0,0.6,0}
\definecolor{mygray}{rgb}{0.5,0.5,0.5}
\definecolor{mymauve}{rgb}{0.58,0,0.82}

\lstset{language=R}

\lstset{ %
  backgroundcolor=\color{white},   % choose the background color; you must add \usepackage{color} or \usepackage{xcolor}; should come as last argument
  basicstyle=\footnotesize,        % the size of the fonts that are used for the code
  breakatwhitespace=false,         % sets if automatic breaks should only happen at whitespace
  breaklines=true,                 % sets automatic line breaking
  captionpos=b,                    % sets the caption-position to bottom
  commentstyle=\color{mygreen},    % comment style
  deletekeywords={...},            % if you want to delete keywords from the given language
  escapeinside={\%*}{*)},          % if you want to add LaTeX within your code
  extendedchars=true,              % lets you use non-ASCII characters; for 8-bits encodings only, does not work with UTF-8
  frame=single,	                   % adds a frame around the code
  keepspaces=true,                 % keeps spaces in text, useful for keeping indentation of code (possibly needs columns=flexible)
  keywordstyle=\color{blue},       % keyword style
  numbers=left,                    % where to put the line-numbers; possible values are (none, left, right)
  numbersep=5pt,                   % how far the line-numbers are from the code
  numberstyle=\tiny\color{mygray}, % the style that is used for the line-numbers
  rulecolor=\color{black},         % if not set, the frame-color may be changed on line-breaks within not-black text (e.g. comments (green here))
  showspaces=false,                % show spaces everywhere adding particular underscores; it overrides 'showstringspaces'
  showstringspaces=false,          % underline spaces within strings only
  showtabs=false,                  % show tabs within strings adding particular underscores
  stepnumber=1,                    % the step between two line-numbers. If it's 1, each line will be numbered
  stringstyle=\color{mymauve},     % string literal style
  tabsize=4,	                     % sets default tabsize to 2 spaces
}

\title{BINFF-402 --- Genomics, Proteomics and Evolution \\ Assignment 1}
\author{Robin Petit}

\begin{document}
\maketitle

\section{Introduction}

The Illumina Infinium 450K sequencer~\cite{infinium450} was designed in order to sequence DNA methylation. It works by first applying
bisulfite transformation to the DNA fragments to transform unmethylated Cytosines to Uracils, and leave methylated Cytosines unchanged,
and then sequencing uses two different bead types: one to determine methylated loci and one to determine unmethylated
loci~\cite{weisenberger2008comprehensive}.

The Infinium HumanMethylation450 BeadChip contains over 4.5e5 methylation sites probes, and two different chemical assays: Infinium I and
Infinium II. In this assignment, only Infinium I is considered. Yet, Infinium II should also be considered in analyses, but
separately~\cite{dedeurwaerder2011evaluation}.

In this document, methylation level is studied in the case of a DKO of genes DNMT1 and DNMT3b (respectively OMIM ids 126375 and 602900)~\cite{amberger2008OMIM}
which are both methyltransferases. This double knockout induces a loss of hypermethylated CpG islands which is not present in case of
single knockout of either of these genes~\cite{paz2003genetic}.

The study has been performed with the R language. Note that all code has been written from scratch even though there is a wide range of
available packages for bioinformatics in R.~\cite{gentleman2004bioconductor} See for instance the package IMA that is made on purpose
for Infinium data analysis.~\cite{wang2012ima}

All the R code used for this report can be found at~\url{https://github.com/RobinPetit/BINF-F402}.

\section{Analysis}
\subsection{$\beta$-value distribution}

\begin{figure*}[!t]
	\begin{subfigure}{.33\textwidth}
	\includegraphics[width=\textwidth]{../plots/plot0.eps}
	\end{subfigure}
	\begin{subfigure}{.33\textwidth}
	\includegraphics[width=\textwidth]{../plots/plot1.eps}
	\end{subfigure}
	\begin{subfigure}{.33\textwidth}
	\includegraphics[width=\textwidth]{../plots/plot2.eps}
	\end{subfigure}
	\begin{subfigure}{.33\textwidth}
	\includegraphics[width=\textwidth]{../plots/plot3.eps}
	\end{subfigure}
	\begin{subfigure}{.33\textwidth}
	\includegraphics[width=\textwidth]{../plots/plot4.eps}
	\end{subfigure}
	\begin{subfigure}{.33\textwidth}
	\includegraphics[width=\textwidth]{../plots/plot5.eps}
	\end{subfigure}
	\caption{Distribution of the $\beta$ value for each sample. Samples 1 to 3 are the controls, and samples 4 to 6 are the cases.
	\label{fig:beta-distribution}}
\end{figure*}

The $\beta$ distribution is shown in~\autoref{fig:beta-distribution}. For each probe, the $\beta$ value is computed to be $\frac M{M+U+\alpha}$,
where $M$ is the methylated score, $U$ is the unmethylated score, and $\alpha$ acts as a pseudo-count to avoid the case $M=U=0$ to lead
to a division by zero. This $\alpha$ value has been set to 100~\cite{du2010comparison}.

It is clear that controls have a two-spikes $\beta$ distribution: most probes have a methylation $\beta$-value around 0 or around 1.
A high methylation $\beta$-value (close to 1) means that almost all of the sequenced cells had a methylated Cytosine at this position,
whereas a low methylation $\beta$-value (close to 0) means that almost none of the sequenced cells had a methylated Cytosine at this
position. Yet, as several different cells are sequenced at the same time, it is possible to have a Cytosine that is methylated in
some cells, but not in others. This explains how $\beta$-value can take so many different values.

It is also clear that cases have a single-spike $\beta$ distribution: most probes have a methylation $\beta$-value close to 0,
but almost none has a $\beta$-value close to 1. Also, there is a higher density of intermediate $\beta$-values, i.e. between $0.2$
and $0.8$, which means that there is still methylation happening in the cases, but the methylation of DNA varies from one cell
to another.

\subsection{Global overview}



% references
\newpage
\bibliographystyle{IEEEtran}
\bibliography{report}{}

% Second part
\newpage
\section{R source code}

First of all, libraries must be imported and constants must be defined. Only two libraries have been used files must be loaded into R:

\begin{lstlisting}
INFINIUM_PATH  <- 'Infinium450k_raw_data.txt';
ANNO_MINI_PATH <- 'HumanMethylation450_anno_mini.csv';
NB_CASES <- 3;
NB_CONTROLS <- NB_CASES;
CONTROLS <- 1:NB_CONTROLS;
CASES <- (1:NB_CASES) + NB_CONTROLS;
ALPHA_PSEUDO_COUNT <- 100;
ALL_CHROMOSOMES <- c(1:22, 'X', 'Y');
CHROMOSOMES_TO_TEST <- 19:21;

# Use first column as row names
infinium <- read.table(INFINIUM_PATH, header=T, dec=',', row.names=1);
annotations <- read.table(ANNO_MINI_PATH, header=T, sep=',', row.names=1);
\end{lstlisting}

In order to determine the $\beta$ value of a sample, let's define a few functions:
\begin{lstlisting}
compute.beta <- function(signalA, signalB) {
	# Note the pseudo count $\alpha = 100$ to avoid dividing by 0
	return (signalB / (signalA + signalB + ALPHA_PSEUDO_COUNT));
}

# Get the column corresponding to the required sample
get.sample.signal <- function(table, sample.id, signalA=T) {
	if(signalA) {
		return (as.vector(table[[2*sample.id-1]]));
	} else {
		return (as.vector(table[[2*sample.id]]));
	}
}

# Returns the $\beta$ values of a given sample
get.beta <- function(infinium, sample.id) {
	return (compute.beta(
		get.sample.signal(infinium, sample.id, T),
		get.sample.signal(infinium, sample.id, F)
	));
}
\end{lstlisting}

so that plotting becomes:
\begin{lstlisting}
plot.beta.distribution.infinium <- function(infinium) {
	for(sample.id in 1:(NB_CASES+NB_CONTROLS)) {
		plot(
			density(get.beta(infinium, sample.id)),
			xlim=c(-.2, 1.2),
			ylim=c(0, 5),
			xlab=TeX('$\\beta$'),
			ylab=TeX('Density of $\\beta$'),
			main=TeX(paste('density of $\\beta$ distribution of sample', sample.id)),
		);
	}
}
plot.beta.distribution.infinium(infinium);
\end{lstlisting}

\end{document}
